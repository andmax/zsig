\hypertarget{example_zpol_8cc-example}{
\section{example\_\-zpol.cc}
}
This is an example of how to use the compute functions aiding the Zernike Polynomials Basis class. \begin{DoxySeeAlso}{See also}
\hyperlink{zpolbasist_8hh}{zpolbasist.hh}
\end{DoxySeeAlso}

\begin{DoxyCodeInclude}

#include <limits>
#include <fstream>
#include <sstream>

#include <zpolbasist.hh>

#define ZERNIKE_ORDER 8
#define ZERNIKE_VALUE_TYPE long double

#define DOMAIN_GRID_X 256
#define DOMAIN_GRID_Y 256

typedef zsig::ZernikePolynomialsBasisT< ZERNIKE_ORDER, ZERNIKE_VALUE_TYPE > zpolb
      asis_type;

typedef unsigned char ppmimg [DOMAIN_GRID_X * DOMAIN_GRID_Y * 3];

void hsv_to_rgb( const ZERNIKE_VALUE_TYPE& h,
                 const ZERNIKE_VALUE_TYPE& s,
                 const ZERNIKE_VALUE_TYPE& v,
                 ZERNIKE_VALUE_TYPE& r,
                 ZERNIKE_VALUE_TYPE& g,
                 ZERNIKE_VALUE_TYPE& b ) {

        if (s == 0) { r = v; g = v; b = v; }
        else {

                ZERNIKE_VALUE_TYPE var_h = h * 6;
                ZERNIKE_VALUE_TYPE var_i = floor( var_h );
                ZERNIKE_VALUE_TYPE var_1 = v * ( 1 - s );
                ZERNIKE_VALUE_TYPE var_2 = v * ( 1 - s * ( var_h - var_i ) );
                ZERNIKE_VALUE_TYPE var_3 = v * ( 1 - s * ( 1 - ( var_h - var_i ) 
      ) );

                if      ( var_i == 0 ) { r = v     ; g = var_3 ; b = var_1; }
                else if ( var_i == 1 ) { r = var_2 ; g = v     ; b = var_1; }
                else if ( var_i == 2 ) { r = var_1 ; g = v     ; b = var_3; }
                else if ( var_i == 3 ) { r = var_1 ; g = var_2 ; b = v;     }
                else if ( var_i == 4 ) { r = var_3 ; g = var_1 ; b = v;     }
                else                   { r = v     ; g = var_1 ; b = var_2; }

        }

}

void scalar_to_rgb( const ZERNIKE_VALUE_TYPE& scalar,
                    ZERNIKE_VALUE_TYPE& red,
                    ZERNIKE_VALUE_TYPE& green,
                    ZERNIKE_VALUE_TYPE& blue ) {

        static const ZERNIKE_VALUE_TYPE value = 90/100.0;
        static const ZERNIKE_VALUE_TYPE saturation = 60/100.0;
        static const ZERNIKE_VALUE_TYPE min_hue = 212/360.0, max_hue = 0.0;

        ZERNIKE_VALUE_TYPE hue = (max_hue - min_hue) * scalar + min_hue;

        hsv_to_rgb( hue, saturation, value, red, green, blue );

}

int main( int argc, char *argv[] ) {

        std::cout << "[zsig] Usage: " << argv[0] << " [write] (where write = 1 ou
      tputs Zernike images as ppm)\n"
                  << "[zsig] Example ** 1 ** Building Zernike Basis\n"
                  << "[zsig] Allocating memory for Zernike Polynomials Basis\n"
                  << "[zsig] The domain grid is: " << DOMAIN_GRID_X << " x " << D
      OMAIN_GRID_Y
                  << " with Zernike Order = " << ZERNIKE_ORDER
                  << " and each value type = " << sizeof(ZERNIKE_VALUE_TYPE) << "
       Bytes\n";

        zpolbasis_type **ZernikeBasis = new zpolbasis_type*[DOMAIN_GRID_X];
        for (unsigned i = 0; i < DOMAIN_GRID_X; ++i)
                ZernikeBasis[i] = new zpolbasis_type[DOMAIN_GRID_Y];

        std::cout << "[zsig] Computing the Zernike Polynomials Basis\n";

        zsig::compute_basis( ZernikeBasis, DOMAIN_GRID_X, DOMAIN_GRID_Y );

        if( argc == 1 ) { std::cout << "[zsig] Done!\n"; return 0; }

        std::cout << "[zsig] Writing images corresponding to the Zernike Basis\n"
      ;

        for (unsigned p = 0; p < ZERNIKE_ORDER; ++p) {

                for (unsigned qi = 0; qi <= p/2; ++qi) {

                        std::stringstream ss;
                        if( p % 2 == 0 ) ss << "Zimg_" << p << "_" << qi*2 << ".p
      pm";
                        else ss << "Zimg_" << p << "_" << qi*2+1 << ".ppm";
                        std::ofstream outppm(ss.str().c_str());

                        std::cout << "[zsig] Making image: " << ss.str() << "\n";
      

                        outppm << "P6\n" << DOMAIN_GRID_X << " " << DOMAIN_GRID_Y
       << "\n255\n";

                        // min/max scalar values
                        ZERNIKE_VALUE_TYPE mins = std::numeric_limits< ZERNIKE_VA
      LUE_TYPE >::max(),
                        maxs = -std::numeric_limits< ZERNIKE_VALUE_TYPE >::max();
      

                        for (unsigned gx = 0; gx < DOMAIN_GRID_X; ++gx) {

                                for (unsigned gy = 0; gy < DOMAIN_GRID_Y; ++gy) {
      

                                        mins = std::min( mins, ZernikeBasis[gx][g
      y][p][qi].real() );
                                        maxs = std::max( maxs, ZernikeBasis[gx][g
      y][p][qi].real() );

                                }

                        }

                        ppmimg ZernikeImage;

                        ZERNIKE_VALUE_TYPE scalar, red, green, blue;

                        for (unsigned gx = 0; gx < DOMAIN_GRID_X; ++gx) {

                                for (unsigned gy = 0; gy < DOMAIN_GRID_Y; ++gy) {
      

                                        scalar = (ZernikeBasis[gx][gy][p][qi].rea
      l() - mins) / (maxs - mins);

                                        scalar_to_rgb( scalar, red, green, blue )
      ;

                                        ZernikeImage[ ((DOMAIN_GRID_Y-1-gy)*DOMAI
      N_GRID_X + gx)*3 + 0 ] = (unsigned char)(red * 255);
                                        ZernikeImage[ ((DOMAIN_GRID_Y-1-gy)*DOMAI
      N_GRID_X + gx)*3 + 1 ] = (unsigned char)(green * 255);
                                        ZernikeImage[ ((DOMAIN_GRID_Y-1-gy)*DOMAI
      N_GRID_X + gx)*3 + 2 ] = (unsigned char)(blue * 255);

                                }

                        }

                        outppm.write( (const char*)&ZernikeImage[0], DOMAIN_GRID_
      X * DOMAIN_GRID_Y * 3 );

                        outppm.close();

                }

        }

        std::cout << "[zsig] Done!\n";

        return 0;

}
\end{DoxyCodeInclude}
 