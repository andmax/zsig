\hypertarget{example_zproj_8cc-example}{
\section{example\_\-zproj.cc}
}
This is an example of how to use the Zernike Polynomials Orthogonal Basis class and auxiliary functions. \begin{DoxySeeAlso}{See also}
\hyperlink{zpolbasist_8hh}{zpolbasist.hh}
\end{DoxySeeAlso}

\begin{DoxyCodeInclude}

#include <limits>
#include <fstream>
#include <sstream>

#include <zpolbasist.hh>

#define ZERNIKE_ORDER 8
#define ZERNIKE_VALUE_TYPE long double

#define DOMAIN_GRID_X 256
#define DOMAIN_GRID_Y 256

typedef zsig::ZernikePolynomialsBasisT< ZERNIKE_ORDER, ZERNIKE_VALUE_TYPE > zpolb
      asis_type;

typedef unsigned char pgmimg [DOMAIN_GRID_X][DOMAIN_GRID_Y];

void river_gray_image( pgmimg& img, const bool& rotated = false ) {

        float x, y, r;
        unsigned char gray;

        for (unsigned gx = 0; gx < DOMAIN_GRID_X; ++gx) {

                x = ( 2.f * gx + 1.f ) / (float)(DOMAIN_GRID_X) - 1.f;

                for (unsigned gy = 0; gy < DOMAIN_GRID_Y; ++gy) {

                        y = ( 2.f * gy + 1.f ) / (float)(DOMAIN_GRID_Y) - 1.f;

                        r = (float)sqrt( x*x + y*y );

                        if( r < 1.f )
                                gray = (unsigned char)( 255 * (rotated?(x*x):(y*y
      )) );
                        else
                                gray = 0.f;

                        img[gx][gy] = gray;

                }

        }

}

void write_pgm( pgmimg& img, const char *fn ) {

        std::ofstream out(fn);

        out << "P5\n" << DOMAIN_GRID_X << " " << DOMAIN_GRID_Y << "\n255\n";

        out.write( (const char*)&img[0][0], DOMAIN_GRID_X * DOMAIN_GRID_Y );

        out.close();

}

int main( int argc, char *argv[] ) {

        std::vector<float[15]> a;

        std::cout << "[zsig] Usage: " << argv[0] << " [write] (where write = 1 ou
      tputs all images as ppm)\n"
                  << "[zsig] Example ** 2 ** Project / Reconstruct / Compare\n"
                  << "[zsig] Allocating memory for Zernike Polynomials Basis\n"
                  << "[zsig] The domain grid is: " << DOMAIN_GRID_X << " x " << D
      OMAIN_GRID_Y
                  << " with Zernike Order = " << ZERNIKE_ORDER
                  << " and each value type = " << sizeof(ZERNIKE_VALUE_TYPE) << "
       Bytes\n";

        bool write_images = ( argc == 2 and argv[1][0] == '1' );

        zpolbasis_type **ZernikeBasis = new zpolbasis_type*[DOMAIN_GRID_X];
        for (unsigned i = 0; i < DOMAIN_GRID_X; ++i)
                ZernikeBasis[i] = new zpolbasis_type[DOMAIN_GRID_Y];

        std::cout << "[zsig] Computing the Zernike Polynomials Basis\n";

        zsig::compute_basis( ZernikeBasis, DOMAIN_GRID_X, DOMAIN_GRID_Y );

        std::cout << "[zsig] Generating river gray image\n";

        pgmimg river;

        river_gray_image( river );

        if( write_images ) {

                std::cout << "[zsig] Writing generated image: river.pgm\n";

                write_pgm( river, "river.pgm" );

        }

        std::cout << "[zsig] Converting it to Zernike coefficients\n";

        ZERNIKE_VALUE_TYPE **f_river = new ZERNIKE_VALUE_TYPE*[DOMAIN_GRID_X];

        for (unsigned gx = 0; gx < DOMAIN_GRID_X; ++gx) {

                f_river[gx] = new ZERNIKE_VALUE_TYPE[DOMAIN_GRID_Y];

                for (unsigned gy = 0; gy < DOMAIN_GRID_Y; ++gy) {

                        f_river[gx][gy] = (ZERNIKE_VALUE_TYPE)( river[gx][gy] / 2
      55.0 );

                }

        }

        zpolbasis_type Zriver;

        Zriver.project( f_river, ZernikeBasis, DOMAIN_GRID_X, DOMAIN_GRID_Y );

        std::cout << "[zsig] Reconstructing image from Zernike coefficients\n";

        for (unsigned gx = 0; gx < DOMAIN_GRID_X; ++gx)
                for (unsigned gy = 0; gy < DOMAIN_GRID_Y; ++gy)
                        f_river[gx][gy] = (ZERNIKE_VALUE_TYPE)0;

        Zriver.reconstruct( f_river, ZernikeBasis, DOMAIN_GRID_X, DOMAIN_GRID_Y )
      ;

        // min/max scalar values
        ZERNIKE_VALUE_TYPE mins = std::numeric_limits< ZERNIKE_VALUE_TYPE >::max(
      ),
        maxs = -std::numeric_limits< ZERNIKE_VALUE_TYPE >::max();

        for (unsigned gx = 0; gx < DOMAIN_GRID_X; ++gx) {

                for (unsigned gy = 0; gy < DOMAIN_GRID_Y; ++gy) {

                        if( f_river[gx][gy] == 0 ) continue;

                        mins = std::min( mins, f_river[gx][gy] );
                        maxs = std::max( maxs, f_river[gx][gy] );

                }

        }

        for (unsigned gx = 0; gx < DOMAIN_GRID_X; ++gx) {

                for (unsigned gy = 0; gy < DOMAIN_GRID_Y; ++gy) {

                        if( f_river[gx][gy] == 0 ) { river[gx][gy] = 0; continue;
       }

                        river[gx][gy] = (unsigned char)( 255 * (f_river[gx][gy] -
       mins) / (maxs - mins) );

                }

        }

        if( write_images ) {

                std::cout << "[zsig] Writing reconstructed river gray image: reco
      n_river.pgm\n";

                write_pgm( river, "recon_river.pgm" );

        }

        std::cout << "[zsig] Generating rotated river gray image\n";

        river_gray_image( river, true );

        if( write_images ) {

                std::cout << "[zsig] Writing generated image: rot_river.pgm\n";

                write_pgm( river, "rot_river.pgm" );

        }

        std::cout << "[zsig] Converting it to Zernike coefficients\n";

        for (unsigned gx = 0; gx < DOMAIN_GRID_X; ++gx)
                for (unsigned gy = 0; gy < DOMAIN_GRID_Y; ++gy)
                        f_river[gx][gy] = (ZERNIKE_VALUE_TYPE)( river[gx][gy] / 2
      55.0 );

        zpolbasis_type Zrotated;

        Zrotated.project( f_river, ZernikeBasis, DOMAIN_GRID_X, DOMAIN_GRID_Y );

        std::cout << "[zsig] Comparing river and rotated Zernike coefficients\n";
      

        ZERNIKE_VALUE_TYPE dist = Zriver.compare( Zrotated );

        std::cout << "[zsig] Give the Euclidean distance of these two image in Ze
      rnike space: " << dist << "\n";

        std::cout << "[zsig] Done!\n";

        return 0;

}
\end{DoxyCodeInclude}
 